\documentclass[12pt]{scrartcl}
\usepackage[utf8]{inputenc}
\usepackage[english,croatian]{babel}
\usepackage[unicode]{hyperref}
\usepackage{amsmath,amssymb,amsthm}
\usepackage{mathtools}
\usepackage{thmtools}
\usepackage{csquotes}
\usepackage[style=numeric]{biblatex}
\usepackage{algorithm}
\usepackage{algpseudocode}
\usepackage{listings}
\usepackage{tikz}
\usetikzlibrary{patterns,decorations.pathreplacing}

\definecolor{codegreen}{rgb}{0,0.6,0}
\definecolor{codegray}{rgb}{0.5,0.5,0.5}
\definecolor{codepurple}{rgb}{0.58,0,0.82}
\definecolor{backcolour}{rgb}{0.95,0.95,0.92}

\lstdefinestyle{mystyle}{
    backgroundcolor=\color{backcolour},   
    commentstyle=\color{codegreen},
    %keywordstyle=\color{magenta},
    keywordstyle=\color{blue},
    numberstyle=\tiny\color{codegray},
    stringstyle=\color{codepurple},
    basicstyle=\ttfamily\footnotesize,
    breakatwhitespace=false,         
    breaklines=true,                 
    captionpos=b,                    
    keepspaces=true,                 
    numbers=left,                    
    numbersep=5pt,                  
    showspaces=false,                
    showstringspaces=false,
    showtabs=false,                  
    tabsize=2
}

\lstset{style=mystyle}
\renewcommand\lstlistingname{Implementacija}
\renewcommand\lstlistlistingname{Implementacije}
\def\lstlistingautorefname{Impl.}

\hypersetup{
    colorlinks,
    linkcolor={red!50!black},
    citecolor={blue!50!black},
    urlcolor={blue!80!black}
}

\MakeOuterQuote{"}
\declaretheorem{teorem}
\declaretheorem[sibling=teorem]{lema}
\declaretheorem[style=definition,sibling=teorem,qed=$\vartriangleleft$]{definicija}
\declaretheorem[name=Primjer,style=definition]{example}

%\newcommand{\citat}[2]{\begin{quotation}\textit{#1}\end{quotation}\vspace{-1em}\begin{flushright}---#2\end{flushright}}
\newcommand{\citat}[2]{\begin{quotation}\textit{#1}\vspace{-1em}\begin{flushright}---#2\end{flushright}\end{quotation}}
\newcommand{\primjer}[2]{%
    \renewcommand\qedsymbol{$\vartriangleleft$}%
    \begin{example}%
        #1%
    \end{example}%
    \begin{proof}[Rješenje]%
        #2%
    \end{proof}%
    \renewcommand\qedsymbol{$\square$}
}

\newcommand{\algoritam}[2]{%
\begin{algorithm}
\floatname{algorithm}{Algoritam}
\caption{#1}
\begin{algorithmic}
#2
\end{algorithmic}
\end{algorithn}
}

\title{Miko --- jezik za mikološke simulacije}
\author{Petar Pavlović i Mauro Raguzin}
\date{\today}

\begin{document}
\maketitle
\tableofcontents
\pagebreak

\section{Opis jezika}
Implementiran je jezik \textbf{Miko} koji omogućuje vođenje "pametnije" baze mikoloških uzoraka od recimo SQL baze. Pritom su implementirane specifične
mogućnosti simulacije mutacije, selekcije i križanja korištenjem jednostavnih genetskih algoritama. Jezik je dizajniran da omogućuje laku manipulaciju
numeričkim podacima te podacima koji predstavljaju uzorke gljiva s punim informacijama o pronalasku, tipu gljive te DNA. Implementacija se sastoji
od interpretera ovog jezika napisanog u Pythonu.

Jezik ima sljedeće mogućnosti:
\begin{itemize}
    \item Rad s listama svih ugrađenih osnovnih tipova: broj (svi su \verb|double| interno), string, bool kao i objekata vezanih za domenu;
    \item Ugnježđene liste i aritmetiku nad njima;
    \item Aritmetika s jedinicama tj.\ dimenzijama; dimenzionalni \textsl{mismatch} je greška pri izvođenju ako se nije mogla dokazati statički, ali 
    interpreter uključuje i statički analizator tipova koji pokuša što je više moguće dokazati pri samom parsiranju i tako unaprijed izbjeći nemoguće
    operacije poput množenja/dijeljenja dimenzionalnih veličina (koje su u ovoj domeni samo mase) ili zbrajanja dimenzijske i nedimenzijske veličine;
    \item Konkateniranje stringova sa $+$;
    \item Korisnički definirane funkcije;
    \item Ugrađene \verb|read| i \verb|write| funkcije koje (de)serijaliziraju iz/u JSON bilo koji objekt koji se koristi u programu;
    \item \verb|print| funkcija s očitom namjenom, kao u Pythonu;
    \item Standardnu \verb|for| petlju na jednoj varijabli;
    \item Grananje kao u C-u;
    \item Podrška za sve operatore (osim \verb|compound assignment|) iz C-a.
\end{itemize}
Jezik je \textsl{dynamically typed}, no sve se varijable moraju deklarirati prije uporabe; redeklaracije nisu dozvoljene te su svi objekti koje stvaramo
imutabilni, osim \verb|Tree| objekata koji predstavljaju taksonomiju pojedine gljive. Ona se naime konstruira s default konstruktorom i onda se
korištenjem dot-operatora (kao u C++-u) može pristupati pojedinim komponentama taksonomije i dodijeliti im stringove. Tijekom parsiranja se obavljaju
mnoge statičke provjere tipa koje pokušaju otkriti što više moguće ilegalnih konstrukata prije izvođenja programa
 poput aritmetike između nekompatibilnih stvari (
    samo neki primjeri: lista i broj, dimenzioniran broj plus nedimenzioniran broj, string plus nestring\ldots). No naravno problem statičkog
određivanja tipa svakog izraza je u ovakvom jeziku općenito neodlučiv pa neće biti pokrivene situacije u kojima se pojavljuju recimo samo varijable (ne izvodimo
daljnju analizu toka kroz funkcije programa kako bismo potencijalno suzili skup mogućih tipova za svaku vidljivu varijablu).

Same gljive tj.\ uzorci su također objekti tipa \verb|Fungus| koji se konstruiraju s \verb|Fungus| konstruktorom; detalji su dani kroz komentare
u izvornom kodu na \href{https://github.com/mraguzin/ip-projekt}{repozitoriju} ovog rada. Glavna funkcionalnost je rad s listama objekata gljiva
koristeći tri genetska operatora: \textbf{selekcija}, \textbf{križanje} i \textbf{mutacija}. Križanje je binaran operator, ostali su unarni i rade
poštujući neke globalne postavke koje korisnik može postavljati koristeći builtin \verb|setParam| funkciju.

\section{Neki primjeri programa}
U datoteci \verb|izoliranitest.py| unutar \verb|package| direktorija repozitorija se (na samom dnu) nalazi niz testnih programa jezika koji
pokazuju neke funkcionalnosti. Primjerice, za vidjeti kako se radi s listama, jedinicama i genetskim operatorima u ovom jeziku, provjerite
zadnja dva primjera u spomenutoj datoteci. Zadnji primjer tamo prikazuje kako bi se u osnovi ovakav jezik mogao koristiti u nekom stvarnom
istraživačkom projektu (tj.\ koristeći te komponente jezika), iako bi naravno još trebalo poraditi na točnom značenju
i djelovanju genetskih operatora te još nekih specifičnih
funkcionalnosti koje bi konkretnim znanstvenicima olakšale svakodnevni rad i učinile ih produktivnijima.

%ISPRAVLJENO!!  &Napomena: postoji greška pri parsiranju mutacije u zadnjem primjeru, na tome još radimo i bit će najvjerojatnije ažurirano negdje do petka ujutro
%i sve bi onda trebalo raditi kako je zamišljeno.



\end{document}