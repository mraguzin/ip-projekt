\documentclass{article}
\usepackage[utf8]{inputenc}
\usepackage[croatian]{babel}
\usepackage{graphicx} % Required for inserting images
\usepackage{listings}
\usepackage[style=numeric]{biblatex}
\usepackage{csquotes}
\MakeOuterQuote{"}

\addbibresource{clanak.bib}

\title{Obrada mikoloških uzoraka}
\author{Mauro Raguzin, Petar Pavlović}
\date{Lipanj 2023}

\begin{document}

\maketitle

\section{Uvod}
U ovom dokumentu istražit ćemo kako se napisani jezik iz Interpretacije programa koristi, koje ima
funkcionalnosti i kako bi se mogao primijenjivati u budućnosti. Jezik se bavi obradom mikoloških uzoraka,
to jest omogućava lako rukovanje karakteristikama gljiva: imenima, latinskim imenima, DNA uzorcima,
taksonomijom i vremenom pronalaska.
\\Jezik podržava neke bazične tipove podataka i operacije, poput brojeva, string imena, vektora,
računskih operacija, pridruživanja vrijednosti, stvljanja komentara te if-else uvjeta i for petlji.
Također je omogućeno baratanje mjernom jedinicom mase, pisanje i čitanje iz datoteka, te stvaranje
vlastititih funkcija.
Ono što ovaj jezik čini specifičnim za posebnu primjenu su genetski operatori na koje ćemo se u ovom
dokumentu malo više osvrnuti. 

\section{Genetski operatori}
DNA sekvenciranje je proces određivanja redoslijeda nukleotida kod neke jedinke. Postoji četiri vrste
nukleotida: timin (T), adenin (A), citozin (C) i gvanin (G). Timin i adenin, te citozin i gvanin uvijek
dolaze "u paru" spojeni jedan na drugoga, tako da je navodenjem nukleotida s jedne strane DNA strukture
cijela sekvenca jedinstveno definirana. Odabrali smo tri genetska operatora: mutaciju, križanje i
selekciju. Ponašanje tih operacija u prirodi je složeno, i njihove prave verzije bi trebao implementirati
neki mikolog koji će koristiti ovaj jezik. Stoga smo se ograničili na neko proizvoljno, ali razumno
ponašanje tih operatora kako bi se pokazala njihova osnovna funkcionalnost na kraćim nizovima u
demonstrativne svrhe.
\subsection{Križanje}
Operator križanja simulira spajanje genetskog materijala. On prima dvije gljive i vraća njihovo genetsko
"dijete". Algoritam nasumično izabire hoće li duljina djetetovog genoma biti kao duljina prve ili druge
gljive, te do duljine kraćeg roditelja nasumično izabire nečiji gen na tom mjestu, a na ostatku uzima gene
dužeg ako je tako izabrano. Također je omogućeno križanje na listama jedinki.
\subsection{Mutacija}
Mutacija je promijena redoslijeda nukleotida u genomu. Ovaj operator prima tip podatka gljiva (Fungus) ili
listu više gljiva i vraća gljive mutoranog genetskog materijala. Algoritam koji je korišten prolazi kroz
DNA sekvencu gljive i zamijeni u prosjeku trećinu nukleotida nekim nasumičnim nukleotidom (možda i istim).
U prirodi se kod mutacije sigurno mijenja puno manje od trećine gena, ali ideja je bila da se u našem
programu uvijek dogodi nekakva promijena ali da se opet vidi sličnost s originalnom gljivom.
\subsection{Selekcija}
Operator selekcije uzima listu jedinki i po nekom kriteriju bira onu s "najboljim" genetskim kodom i nju
vraća. Kriterij koji je izabran je genetska raznolikost, dakle bit će izabrane gljive koje nemaju
veliku razliku u broju različitih nukleotida. To se postiže tako da se računa zbroj kvadrata svih razlika
između bojeva nukleotida. Istina je da priroda cijeni i nagraduje genetsku raznolikost, ali naravno da je
u praksi to kompleksniji algoritam.

\section{Primjeri programa}
\textbf{Pogledajmo prvo primjere nekih mogućnosti koje jezik nudi.}
\subsection{Prvi program}
\begin{lstlisting}
program1 = """
let var := "nesto";
if (var) {
   var := "novo";
}
print (var);
var := 5;
let accum := 0;
for var {
   accum := accum + 2;
   print (accum);
}
let v1 := [3, 54g, 200mg, "prvi i ", -1];
let v2 := [1, 1g, 1g, "drugi", 0];
print(v1+v2);
let datum := 1.1.2011.;
let vrijeme := 13.9.2013. 10:10;
print (datum);
print (vrijeme);
let ne := Number("14.22"); # konvertirajuci konstruktor
print(ne);
"""
P(program1).izvrsi()
\end{lstlisting}
\textbf{Rezultat ovog programa:}
\begin{lstlisting}
novo
2.0
4.0
6.0
8.0
10.0
[4.0, 55.0 g, 1200.0 mg, prvi i drugi, -1.0, ]
1.1.2011.
13.9.2013. 10:10:0
14.22
\end{lstlisting}
\subsection{Drugi program}
\textbf{Kako funkcioniraju funkcije i datoteke:}
\begin{lstlisting}
program2 = """
function mojafun(a, b) {
  let null;
  if(a) {
    null := "a je istinit";
  }
  else {
    null := 0;
  }
  return null;
}
write("datoteka1.txt",mojafun(true, 0));
write("datoteka2.txt",mojafun(false, 0));
let ucitano1 := read("datoteka1.txt");
let ucitano2 := read("datoteka2.txt");
print(ucitano1);
print(ucitano2);
"""
P(program2).izvrsi()
\end{lstlisting}
\textbf{Rezultat ovog programa:}
\begin{lstlisting}
a je istinit
0.0
\end{lstlisting}
\subsection{Treći program}
Prikažimo rad sa gljivama. Svaka gljiva ima \textbf{ime}, \textbf{latinsko ime}, \textbf{DNA uzorak}, \textbf{taksonomiju} i opcionalno
\textbf{vrijeme pronalaska}. Ako je vrijeme pronalaska nenavedeno, samo se stavlja trenutno vrijeme.
\begin{lstlisting}
program15 = """
let dna1 := DNA(ATCGGTACG);
let dna2 := DNA(GCCGGTCCG);
let tax := Tree();
tax.phyl := "neka filogeneza";
tax.king := "neko kraljevstvo";
let datum := 25.3.2008.;
let fung1 := Fungus("naziv", "latinski", dna1, tax, datum);
let fung2 := Fungus("naziv2", "latinski2", dna2, tax);
print(fung1);
print(fung2);
"""
p3 = P(program15)
p3.izvrsi()
\end{lstlisting}
\textbf{Rezultat:}
\begin{lstlisting}
Name: naziv
Latin name: latinski
DNA: ATCGGTACG
Taxonomy: 
phylum: neka filogeneza
kingdom: neko kraljevstvo
Time found: 25.3.2008.

Name: naziv2
Latin name: latinski2
DNA: GCCGGTCCG
Taxonomy: 
phylum: neka filogeneza
kingdom: neko kraljevstvo
Time found: 18.6.2023. 12:56:19

\end{lstlisting}
\subsection{Genetski operatori}

\begin{lstlisting}
program4 = """
let dna := DNA(AAAAAAAAAAA);
let tax := Tree();
tax.king := "Fungi";
tax.class := "Agaricomycetes";
tax.ord := "Agaricales";
tax.fam := "Amanitaceae";
tax.gen := "Amanita";
tax.spec := "A. muscaria";

let gljiva := Fungus("muhara", "Amanita muscaria", dna, tax);
let gljiva2 := Fungus("muhara2", "Amanita muscaria", dna2, tax);

let mutirana := mutate gljiva;
let mutirana2:= mutate mutirana;
let dijete := mutirana cross mutirana2;
let najbolja := [gljiva, mutirana, mutirana2]select;
print(gljiva);
print("mutirana");
print(mutirana.dna);
print("mutirana2");
print(mutirana2.dna);
print("dijete");
print(dijete.dna);
print("najbolja");
print(najbolja.dna);
"""
P(program4).izvrsi()
\end{lstlisting}
\textbf{Rezultat:}
\begin{lstlisting}
Name: muhara
Latin name: Amanita muscaria
DNA: AAAAAAAAAAA
Taxonomy: 
species: A. muscaria
genus: Amanita
family: Amanitaceae
order: Agaricales
klasa: Agaricomycetes
kingdom: Fungi
Time found: 18.6.2023. 14:22:52

mutirana
AAATAAATAAA
mutirana2
CATTAACTAAA
dijete
CATTAAATAAA
najbolja
CATTAACTAAA
\end{lstlisting}
\section{Budućnost}
Ovdje prikazan jezik je tek početak, kostur nečega što bi moglo stvarno biti korisno ljudima koji se ovom
temom bave profesionalno. Kada bi se uvela još neka poboljšanja i, naravno, genetski operatori učinili
realističnima kao u prirodi, ovaj jezik stvarno bi mogao poslužiti u znanstvene svrhe. Imati jezik
prilagođen specifičnoj temi olakšalo bi rad jer detalji implementacije i sam programski jezik na kojem je
jezik baziran postaju nebitni, sve je konkretno usmjereno na područje znanstvenog rada.


Područje genetike još je uvijek velikim dijelom nepoznato područje otvoreno novim znanstvenim spoznajama,
pa bi ovakav alat sigurno našao svoju svrhu. Bilo bi također moguće proširiti jezik i na neke biljke i
životinje i tako općenitije istraživati genetiku, simulirati procese u prirodi i uspoređivati simulacije sa
stvarnošću. Kada bi se genetski operatori s vremenom učinili dovoljno sličnima onima u prirodi, možda bi
bilo moguće čak i pokušati simulirati evoluciju neke vrste. Zvuči ambiciozno, i možda bi to bilo
ipak neizvedivo, ali ako bi bilo moguće, ovaj jezik bi sigurno u tome pomogao.\nocite{*}

\printbibliography
Ova knjiga bi bila korisna pri daljnjem istraživanju mogućih smjerova razvoja ovog jezika. Za sada su korištene samo osnovne ideje triju
genetskih operatora, ne i specifičnih detalja njihove implementacije iz knjige.

\end{document}